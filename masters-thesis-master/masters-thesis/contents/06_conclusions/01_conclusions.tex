\chapter{Summary and Conclusions}
\label{chap:conclusions}
This work is only the tip of the iceberg of a broader discussion regarding ethics and implications in the game industry as it is today. We tried to observe and identify which are the critical points that game companies need to consider when developing a game using the F2P business model. We have seen that the situation is changing not only for players but also for game designers. We analyzed some of the possible threats that are emerging today in the game industry. We described the risk of addiction and gambling that the reward systems as well as the marketing and the monetization strategies are affecting the players. We believe that in a perfect world, the charge of a game should be more equally divided between all the players rather than focused only on few "whale" players. Furthermore, we reported a lack of measures and carefulness to protect kids and vulnerable players. It is time to start a serious discussion about in app purchases and advertisement, especially for kid's app. We illustrated how game companies are addressing the servitization switch, selling services instead of products. We noted that the way game companies make money nowadays is not trough the the game itself, but rather trough additional services and products. Finally, we provided ethical considerations and point of discussion that can at least alert players and game companies of what is happening.
Our objective is not to give the final answer and neither to decide what is ethical do and what is not. We believe that this considerations must be addressed by laws and governments that regulate each activity and subsequently by the individual players and the individual game producers. 

That said, the "boiling frog" fable is the perfect methapore of the situation in which we are today. The story tells that if a frog is suddenly put into boiling water, it will immediately jump out, however, if the frog is put in cold water which is gradually and slowly brought to boil, the frog will not perceive the threat and it will be cooked to death. Therefore, if we gradually accept to reduce ethics in favor of profits and we do not put a limit in what is right to do and what it is not, we will soon be in a situation from which it will not be easy to go back.