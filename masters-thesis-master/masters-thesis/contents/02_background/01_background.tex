\chapter{Player Implications}
\label{chap:background}
\section{Pay-to-Win}
The F2P business model arises many implications on players and in this chapter we discuss and analyze the most relevant.
A first implication is when a player can purchase game content that allows him to gain a significant and unfair advantage over the non paying players. This is informally called "Pay-to-Win" effect. Paywalls, is the term used to describe the points where, if a player does not pay, the advancement in the game is stopped. It happens especially in multiplayer games and can create frustration or disappointment in the non paying players leading them to abandon the game or turn into paying users. Pay-to-Win games are usually criticized both from players but also from game designer. They usually connect this phenomena to an unfair and not fun game \cite{alha_free--play_2014}. 
A common approach to avoid that in-game purchases affects gameplay is to allow players to buy only "cosmetic" items e.g. skins, avatars, emotes, themes. League of Legends (Riot Games 2009) and Dota 2 (Valve Corporation 2013) are only two examples of games that adopt this approach. Other games, like Clash of Clans (Supercell 2012), even if allows non-paying players to acquire and collect any game content, due to time constraints, becomes impracticable to play the game competitively without paying. A common justification of Pay-to-Win publishers is: “anything you can buy with real money, can be obtained by simply playing the game”, but even if this is usually true, the time or effort required by the player to obtain the same content of a paying user is usually unfeasible. Nevertheless, the opinion of the players can sometimes make the difference. As an example, when the beta of the game Star Wars Battlefront (EA DICE 2017) was released, the fans strongly protested about the new game progression system and convinced the company to entirely remove the microtransactions system from the game just few hours before its worldwide launch \cite{kain_ea_????, tassi_ea_????}.
\section{Whales} 
From an ethical point of view, one of the most important and critical phenomena that the F2P business model has generated, is the so called "Whale". This concept refers to those people who spend lot of money with in-app purchases and microtransactions. The term was first used by salesmen referring to big clients and also by casinos to refer to big spenders. A statistic reported on Forbes \cite{tassi_why_????}, showed that only 0.15 percent of players contribute to 50 percent of the revenues. Similarly, the website \textit{Tapjoy} \cite{_infographic:_????} has analyzed the habits of players of F2P games, and calculated that whales make on average 7.4 in-app purchases per month, for a median average revenue per paying user (ARPPU) of 335 \$. To give a concrete example of this phenomena, in 2012, at the Game Developers Conference, the 5th Planet chief executive Robert Winkler, announced \cite{_what_2013} that in the game Clash of the Dragons, 40 percent of revenues came only from 2 percent of the gamers spending 1.000 \$ or more. But who are the whales? A demographic report \cite{good_who_????} showed that two third of the whales are male with an average age of 30 years old. Furthermore, it analyzed the time spent on a game and emerged that the time spent by whales is grather by a factor of three compared to other players. Finally, they showed that smartphones and tablet generate almost half of the whales in the whole game industry. For a deeper analysis on whales behaviour we refer to \cite{_how_2015}. 

In the app store there are games that are clearly developed with the idea of whales in mind. Whenever we find a game which makes virtually impossible to keep on playing without paying is a clear sign that game designers want to make us spend. We need to remember that even video games are an industry and as any other industry the main objective is to make money. However, there are some peculiarities that makes game profits morally questionable. There is no clear amount of money to distinguish whales from normal paying players but the term usually refers to people who spends hundreds or thousands of dollars with in-game purchases. Even if everyone has the right to do whatever they wants with their money, the average whale is not a rich person with lot of money to spend but he is usually a common person that decides to spend most of his money in a specific game. Also, the marketing strategy of trying to take as much money as possible from a small subset of players, in order to cover for the non paying players, creates some ethical considerations. 

Finally, we need to consider that buying game content is not the same as buying a normal product. When a player buys game content, he does not own the content, he only own the right to see it. He can not trade, re-sell, or donate it. Furthermore, the purchase is available till the game is alive. But what becomes of all the purchases if the game is shut down?  

\section{Addiction}
Also, as other forms of entertainment, games can create addiction. Especially if we consider F2P games, created with the goal of retaining users as long as possible, exploiting those human mechanism weaknesses which can lead to addiction. Games addiction is a serious problem that can involve people of different ages and can determine social isolation, low imagination, mood alterations and exclusion of real life events in favour of virtual achievements or rewards. Many games rely on a cycle of activities that the player has to continue to perform and that is called "compulsion loop". A game reward can generate the neurological reaction of releasing dopamine, making the player feel satisfied. A very similar reaction occurs with gambling addiction. An even more serious problem happen when a player finds these satisfactions only in a virtual game and do not feel the same emotions in the real world. We need to say that encouraging a deeper and deeper engagement is not only a problem of games but concern every entertainment activity like watching movies or others forms of media.

A peculiarity is that, if we look at the top highest grossing mobile games we can see that most of them offers poker, slot dice games or others games like in a real casino. Is quite obvious to think that they are making money from gambling, indeed, they are just games. Despite the fact that they offer in app purchases to fund turns, they do not offer any real money pay-out. Players can play these "games" just for fun. Double Down Casino (Double Down Interactive 2012) is an example of such a game and it offers prices only with virtual currency. People play only for fun or competition with friends. 

The presence of virtual currency in most of the F2P games that are available today, from a psychological point of view, creates an issue. For some players, earning, managing and spending virtual money, are carried out with the same criteria and feelings we use for real money. Prof. Mark D. Griffiths, a psychologist and expert of games, is famous for his research about game and addiction. He argued that, there are social games that contains gambling elements even if they do not involve real money. As an example, FarmVille (Zynga 2009), a farming simulation social game, contains the same principles of excitement that is generated by gambling, even if it does not involve real money. What the Professor identifies as a core element in encouraging gambling behaviour is "the unpredictability of winning or getting other types of intermittent rewards. Small unpredictable rewards lead to highly engaged and repetitive behavior. In a minority of cases, this may lead to addiction." \cite{kuss_internet_2012}. Nowadays, video games might be free to play, however they are expensive in frustration or wasted time.  Finally, it seems that some "games" available for free on the app store, are not proper games but they are more like profit generators with addictive reward systems. Meaning that they are built with the only purpose of generating revenues without considering players satisfaction or entertainment. Of course, not all the F2P games are like this and if applied correctly the F2P business model can be a win-win both for developers and for players. 



\section{Children}
If an adult decides to spend a lot of money in a game, it's his choice, but when a kids spends significant amount of money, it is a different problem. First of all, because kids do not have the concept of money but also because, nowadays, F2P games are made to encourage and promote in game purchases in several forms. Children cannot determine if a purchase is worth its cost. Of course parents have a big responsibility trying to keep their credit cards or accounts secured, however games, sometimes do not provide an easy to use method to separate the money purchases with the rest of the game content and it is becoming every day more simple to buy in-game content. Occasionally, we read news about kids that spent thousands of dollars in games. To give some example of real stories like this, recently, a 7-years-old, in England, during olny six days spent 6.000 \$ with his dad's credit card, buying upgrades for his virtual Jurassic world park and dinosaurs. Probably the most incredible story regards a Belgian teenager that in three months, charged his parent's credit card with more than 46.000 \$ for game content in the Game of War: Fire Age game \cite{kelly_kids_????}. There are hundreds of stories like these and every day new parents are victims of these types of behavior. 

Another ethical issue regards the prices. On the App Store there are several kid's app with offers or deals as high as 69.99 \$ at a time. The problem does not concern only in app purchases but also advertisement. It happens, that with just few clicks and without realizing it, kids subscribe to weekly or monthly subscription services. 



